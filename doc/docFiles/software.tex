\chapter{Software}

\section{Use Case diagramm}
\section{Klassendiagramm}

\section{Datenbank}


\section{Programm Plattform}

Es gibt prinzipiell zwei Möglichkeiten. Pro Betriebssystem wird eine App entwickelt oder es wird mit einer Cross Plattform gearbeitet\cite{appEinf}.


\begin{tabularx}{\textwidth}{lX}
\textbf{individuell}\cite{appEinf}&\\
iOS & mit  Objective-C (oder C/C++)\\
Android & Java \\
\textbf{Cross Plattform}\cite{crossPlat}&\\
    RhoMobile & This is a solution that uses Ruby, especially loved by Ruby on Rails developers. (Free only for noncommercial applications, prices vary)\\
    Appcelerator & This is a solution that allows you to develop native apps with HTML/Javascript (run through a UIWebView on iPhone) . (Free)\\
    PhoneGap & Similar to Appcelerator, I mentioned these two as they seem to have the most vibrant communities, and most extensive support. (Free)\\
    WidgetPad & are good cross platform development tools. Out of these I would rather prefer Phonegap for iOS and Android Development.\\
\end{tabularx}
\centering{{$\vdots$}}

