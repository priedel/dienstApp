\section{Software}

\subsection{Versionsverwaltung}
Es wird mit GIT gearbeitet (Turtorial\footnote{\url{https://www.kernel.org/pub/software/scm/git/docs/gittutorial.html}})und die Daten liegen auf git-hub{Repository\footnote{\url{https://github.com/priedel/dienstApp}}}. Das Projekt mit \enquote{git clone https://github.com/priedel/dienstApp.git} holen.
 Schreibrechte bei Philipp Riedel (E-Mail\footnote{\href{mailto:riedelp@student.ethz.ch}{riedelp@student.ethz.ch}}) beantragen.

\subsection{Programm Plattform}

Es gibt prinzipiell zwei Möglichkeiten. Pro Betriebssystem wird eine App entwickelt oder es wird mit einer Cross Plattform gearbeitet\cite{appEinf}.


\begin{tabularx}{\textwidth}{lX}
\textbf{individuell}\cite{appEinf}&\\
iOS & mit  Objective-C (oder C/C++)\\
Android & Java \\
\textbf{Cross Plattform}\cite{crossPlat}&\\
    RhoMobile & This is a solution that uses Ruby, especially loved by Ruby on Rails developers. (Free only for noncommercial applications, prices vary)\\
    Appcelerator & This is a solution that allows you to develop native apps with HTML/Javascript (run through a UIWebView on iPhone) . (Free)\\
    PhoneGap & Similar to Appcelerator, I mentioned these two as they seem to have the most vibrant communities, and most extensive support. (Free)\\
    WidgetPad & are good cross platform development tools. Out of these I would rather prefer Phonegap for iOS and Android Development.\\
	Xamarin \footnote{\url{http://www.phoronix.com/scan.php?page=news_item&px=MTUxMzA}}& Xamarin and Microsoft announced a global partnership that enables Microsoft developers to create native mobile Windows, iOS and Android apps with the language they know, C\#, and the tools they love, Visual Studio. This groundbreaking partnership empowers one of the world’s largest developer communities to become the most productive and innovative mobile developers – almost overnight.\\
	\multicolumn{2}{l}{\textbf{WEB mit Schnittstelle zu App oder Native}}\\
	GWT Project \footnote{\url{http://www.gwtproject.org/}} &
	GUI mit GWT zu machen und dann weitere Schnittstellen zu den Spezifischen App zu definieren. Oder GWT Native anzuwenden\\
\end{tabularx}
\centering{{$\vdots$}}

\subsection{Use Case diagramm}
\subsection{Klassendiagramm}

\subsection{Datenbank}

